%
% PPT TeX模板
%
% 中文支持方案: XeTeX + xeCJK
% author: Tony Bai
% 
% compile: 
%     xelatex ppt_gbk.tex
%

\documentclass{beamer}

%\XeTeXinputencoding "GBK"

\usepackage{fontspec}                           % for XeTeX
\usepackage{xunicode}
\usepackage{xltxtra}

%\usepackage{beamerthemesplit}
\usepackage{hyperref}

% xeCJK设置
\usepackage[slantfont, boldfont, CJKaddspaces]{xeCJK}

% view the font list through the cmd "fc-list :lang=en"
\setmainfont{Times New Roman}                               % for normal and italic en
\setsansfont{Arial}
\setmonofont{Courier New}

% view the font list through the cmd "fc-list :lang=zh"
\setCJKmainfont{WenQuanYi Micro Hei}
\setCJKsansfont{WenQuanYi Micro Hei}
\setCJKmonofont{WenQuanYi Micro Hei}
%\setCJKfamilyfont{song}{WenQuanYi Micro Hei}

\XeTeXlinebreaklocale "zh"  
\XeTeXlinebreakskip = 0pt plus 1pt

\usetheme{boxes}

\title{\XeTeX~PPT模板\\采用GBK编码}
\author{Tony Bai\footnote{\url{http://bigwhite.blogbus.com}}}
\date{\today}

\begin{document}

\begin{frame}
\titlepage
\end{frame}

\begin{frame}
\frametitle{Outline}
\tableofcontents
\end{frame}

\begin{frame}
\frametitle{first frame}
\framesubtitle{usage of itemize}
    This is the first frame using \XeTeX~and beamer.
    \begin{itemize}
        \item<1-> xx
        \item<2-> yy
        \item<3-> zz
    \end{itemize}
\end{frame}

\begin{frame}
\frametitle{second frame}
\framesubtitle{usage of enumerate}
    This is the second frame.
    \begin{enumerate}
        \item<1-> xx
        \item<2-> yy
        \item<3-> zz
    \end{enumerate}
\end{frame}

\begin{frame}
\frametitle{third frame}
\framesubtitle{usage of block}
    This is the third frame.
    \begin{block}{Advantage}
        The obvious disadvantage of this approach is that you have to know LaTeX in order to use Beamer. 
    \end{block}
    \begin{block}{disadvantage}
        The advantage is that if you know LaTeX, you can use your knowledge of LaTeX also when creating a presentation, not only when writing papers.
    \end{block}
\end{frame}
\end{document}
